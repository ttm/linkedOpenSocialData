\documentclass[review]{elsarticle}

\usepackage{hyperref}
\usepackage{multirow}
\usepackage{pgfplots}
\usepackage{float}
\usepackage{amssymb}
\usepackage{cleveref}
\usepackage[english]{babel}
\usepackage[utf8]{inputenc}
\usepackage[T1]{fontenc}
%
\pgfplotsset{compat=1.5}
\pgfplotsset
{
	width=0.5\textwidth,
	x tick label style={/pgf/number format/1000 sep=},
  enlarge x limits = 0.0,
  ymajorgrids=true,
	major tick style={draw=none},
  ymin = 0.0,
	every axis/.append style={
		every x tick label/.append style={font=\tiny},
    every y tick label/.append style={font=\tiny},
    every axis label/.append style={font=\small},
    height=37mm,
    width=37mm,
    title style={at={(0.5,0.90)}, font=\normalfont},
    xticklabel style={yshift=4pt}
	}
}

%% `Elsevier LaTeX' style
\bibliographystyle{elsarticle-num}
%
\makeatletter
\def\ps@pprintTitle{%
    \let\@oddhead\@empty
    \let\@evenhead\@empty
    \def\@oddfoot{}%
\let\@evenfoot\@oddfoot}
\makeatother
\begin{document}
%
\begin{frontmatter}
%
\title{Open Linked Social Data for Scientific Benchmarking}
%
\author[pwr]{Renato Fabbri\corref{corresponding}\fnref{kio-url}}
\ead{fabbri@usp.br}
%
\author[pwr]{Osvaldo Novais de Oliveira Junior\fnref{kio-url}}
\ead{chu@ifsc.usp.br}
%
\cortext[corresponding]{Corresponding author}
\address[pwr]{S\~ao Carlos Institute of Physics, S\~ao Paulo
University, Brazil}
%
\fntext[kio-url]{\textit{URL:} \url{http://www.ifsc.usp.br/}}
%
\begin{abstract}
The field of social network analysis and the topic of complex networks
are widely researched.
Recently, a myriad of results have been reported which are based in
diverse datasets most often not accessible to other researchers.
This work exposes an open dataset with diverse provenance and oriented
to provide the scientific community a friendly and common repertoire.
Current data was obtained from Facebook, Twitter, IRC, Email and the specific
instances of ParticipaBR, AA and Cidade Democr\'atica.
These were translated to linked data format to homonenize access,
conform to current best practices and ease analyzes which integrate third
party and provided instances.
This document presents an outline and overall statistics of given
dataset which should favor subsequent work.
\end{abstract}
%
\begin{keyword}
Benchmark Data, Facebook, Twitter, IRC, Email, Complex Networkx
%Hierarchy of Clusters \sep HoC \sep Benchmark Dataset \sep Benchmark Data Generator \sep Artificial Data \sep Cluster Analysis \sep Tree Structured Stick Breaking Process \sep TSSB \sep ...
\end{keyword}

\end{frontmatter}

\section{Introduction}
The enormity of the digital data makes a rapid development of various methods of data analysis,
giving an opportunity to analyze the data from a different perspectives.

\subsection{Benchmarking in the analysis of complexity}
%
Every designer of a new method wants to know how good the proposed method is in comparison with others.
Usually such comparison is made by run different methods on some commonly used benchmark datasets and use some of the proposed evaluation measures, e.g.,~\cite{Dreiseitl200128,Cooper1997107,Douglas2011544,Kampichler2010441,DBLP:journals/corr/abs-1301-7401,steinbach00comparison,Kakkar2014}.
%
% karate club, whatelse?

\section{Materials: data from diverse provenance}

\section{Methods: linked open data}
\subsection{RDF}
\subsection{Data-driven ontology synthesis}

\section{Results: data outline}
\label{outline}

\section{Conclusions}
\label{conclusions}

\section*{References}
%
\bibliography{paper}
%\bibliography{myLastBibfile.bib}
%
\end{document}
